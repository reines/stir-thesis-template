% **************************************************************************************************************
% A Classic Thesis Style
% An Homage to The Elements of Typographic Style
%
% Copyright (C) 2011 Andr'e Miede http://www.miede.de
%
% Adapted some more (:P) by Jamie (jrf) 2012
% Adapted for general use by Claire (cma) 2011
% Ripped out Lyx by Jamie (jrf) 2011
% Adapted to Stirling for Lyx by Jesse (jmb) 2010
%
% License:
% This program is free software; you can redistribute it and/or modify
% it under the terms of the GNU General Public License as published by
% the Free Software Foundation; either version 2 of the License, or
% (at your option) any later version.
%
% This program is distributed in the hope that it will be useful,
% but WITHOUT ANY WARRANTY; without even the implied warranty of
% MERCHANTABILITY or FITNESS FOR A PARTICULAR PURPOSE.  See the
% GNU General Public License for more details.
%
% You should have received a copy of the GNU General Public License
% along with this program; see the file COPYING.  If not, write to
% the Free Software Foundation, Inc., 59 Temple Place - Suite 330,
% Boston, MA 02111-1307, USA.
%
% **************************************************************************************************************
% Note:
%    * You must not use "u etc. in strings/commands that will be spaced out (use \"u or real umlauts instead)
%    * New enumeration (small caps): \begin{aenumerate} \end{aenumerate}
%    * For margin notes: \graffito{}
%    * Do not use bold fonts in this style, it is designed around them
%    * Use tables as in the examples
%    * See classicthesis-ldpkg.sty for useful commands
% *****************************************************************************************************

\documentclass[oneside,titlepage,fleqn,numbers=noenddot,headinclude,10pt,a4paper,footinclude=true,cleardoublepage=empty,abstractoff]{scrreprt}

\newcommand{\myTitle}{University of Stirling PhD Thesis Template\xspace}
\newcommand{\myDegree}{Doctor of Philosophy\xspace}
\newcommand{\myName}{My Name\xspace}
\newcommand{\myDepartment}{Institute of Computing Science and Mathematics\xspace}
\newcommand{\myUni}{\protect{University of Stirling}\xspace}
\newcommand{\myTime}{July 2011\xspace}

% Packages with options that might require adjustments
\usepackage[utf8]{inputenc}
\usepackage[british]{babel}
\usepackage[square,numbers]{natbib}
\usepackage[fleqn]{amsmath}
\usepackage{style/classicthesis-ldpkg}
\usepackage{wrapfig}
\usepackage{listings}
\usepackage[colorinlistoftodos]{todonotes}
\usepackage[eulerchapternumbers,drafting,listings,subfig,beramono,eulermath,dottedtoc,floatperchapter]{style/classicthesis}

% These todos are a really useful way to make notes while you write - consider adding
% your own, such as for adding questions for your supervisor when proofreading etc.
% E.g. \mario{what do you think about this?}
% Also automatically generates a todo list before your table of contents...
\newcommand{\addref}{\todo[inline, color=green!80, size=\footnotesize]{\textbf{Add Reference}}}
\newcommand{\addex}{\todo[inline, color=green!80, size=\footnotesize]{\textbf{Add Example}}}
\newcommand{\pastetext}[1]{\todo[inline, color=blue!80, size=\footnotesize]{\textbf{Paste Text:} #1}}
\newcommand{\addtext}[1]{\todo[inline, color=orange!80, size=\footnotesize]{\textbf{Add Text:} #1}}
\newcommand{\addimage}[1]{\missingfigure[size=\footnotesize]{\textbf{Add Image:} #1}}
\newcommand{\proofread}{\todo[inline, color=purple!80, size=\footnotesize]{\textbf{Proof Read}}}

% Include the list of acronyms
\newcommand{\listofacronymsname}{List of Acronyms}

\newcommand{\listofacronyms}{
	\chapter*{\listofacronymsname}
	\label{front:acronyms}

	\begin{acronym}
		% Define your acronyms here - \acro{label}[acronym]{description}
	\end{acronym}
}


\begin{document}

\frenchspacing
\raggedbottom
\selectlanguage{british}
\pagestyle{plain}


% Frontmatter
\pagenumbering{roman}

\pdfbookmark[0]{Title}{titlepage}

\begin{titlepage}
	% if you want the titlepage to be centered, uncomment and fine-tune the line below (KOMA classes environment)
	%\begin{addmargin}[-20mm]{0mm}
    \begin{center}
        \Large

        \hfill

        \vfill

        \begingroup
            \color{CSMGreen}\spacedallcaps{\myTitle} \\ \bigskip
        \endgroup

        \spacedlowsmallcaps{\myName}

        \vfill

		\large
        \ldots for the Degree of \myDegree \\ \medskip
        \myDepartment \\
        \myUni \\ \bigskip

        \myTime

        \vfill

    \end{center}
  %\end{addmargin}
\end{titlepage}

\refstepcounter{dummy}

\cleardoublepage
\pdfbookmark[0]{Declaration}{declaration}

\chapter*{Declaration} % [final][0]
\label{front:declaration}

I, \myName, hereby declare that this work has not been submitted for any other degree at this University or any other institution and that, except where reference is made to the work of other authors, the material presented is original.

\bigskip \bigskip

\noindent\textit{\myTime}

\bigskip \bigskip

\begin{flushright}
    \begin{tabular}{m{5cm}}
        \\ \hline
    \end{tabular}
\end{flushright}
\cleardoublepage
\pdfbookmark[1]{Abstract}{Abstract}

\chapter*{Abstract}

Short summary of the contents in English\dots
\cleardoublepage
\pdfbookmark[0]{Acknowledgements}{acknowledgements}

\chapter*{Acknowledgements} % [0][0]
\label{front:acknowledgements}

\addtext{acknowledgements}
\cleardoublepage
\pdfbookmark[1]{Publications}{publications}
\chapter*{List of Publications}

\bigskip

\noindent Put your publications from the thesis here. The packages \texttt{multibib} or \texttt{bibtopic} etc. can be used to handle multiple different bibliographies in your document.
\cleardoublepage
\setcounter{tocdepth}{2} % <-- 2 includes up to subsections in the ToC
\setcounter{secnumdepth}{3} % <-- 3 numbers up to subsubsections

%*******************************************************
% Table of contents
%*******************************************************

\refstepcounter{dummy}
\pdfbookmark[1]{\contentsname}{tableofcontents}

\tableofcontents
\clearpage

%*******************************************************
% List of Figures
%*******************************************************    

\refstepcounter{dummy}
\pdfbookmark[1]{\listfigurename}{lof}

\listoffigures
\clearpage

%*******************************************************
% List of Tables
%*******************************************************

\refstepcounter{dummy}
\pdfbookmark[1]{\listtablename}{lot}

\listoftables
\clearpage
\cleardoublepage
\pdfbookmark[0]{Terminology}{terminology}

\chapter*{Terminology} % [0][n]

\addtext{define any special terminology}

\cleardoublepage


% Content
\pagenumbering{arabic}
\acresetall % Reset all acronyms so their full form is re-used within content

\chapter{Chapter 1}

\chapter{Chapter 2}

\chapter{Chapter 3} % [0][10]


\chapter{Chapter 4} % [0][10]


\chapter{Chapter 5}

\chapter{Chapter 6}

\chapter{Chapter 7} % [0][10]




% Appendices
\pagenumbering{arabic}
\appendix

\chapter{Appendix A} % [0][0]

\cleardoublepage


% Bibliography
\pagenumbering{roman}

\pdfbookmark[0]{Bibliography}{bibliography}

\bibliographystyle{plainnat}
\bibliography{references}
\cleardoublepage

\end{document}

